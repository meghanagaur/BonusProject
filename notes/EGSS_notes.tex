%--------------------
% Packages
% -------------------
\documentclass[11pt,a4paper]{article}
\usepackage[utf8x]{inputenc}
\usepackage[T1]{fontenc}
\usepackage{color}
%\usepackage{gentium}
\usepackage{mathptmx} % Use Times Font

\usepackage{amsthm}

\usepackage[pdftex]{graphicx} % Required for including pictures
\usepackage[swedish]{babel} % Swedish translations
\usepackage[pdftex,linkcolor=black,pdfborder={0 0 0}]{hyperref} % Format links for pdf
\usepackage{calc} % To reset the counter in the document after title page
\usepackage{enumitem} % Includes lists
\newtheorem{theorem}{Theorem}[section]
\newtheorem{corollary}{Corollary}[theorem]
\newtheorem{lemma}[theorem]{Lemma}
\newtheorem{assumption}[theorem]{Assumptions}
\frenchspacing % No double spacing between sentences
\linespread{1.2} % Set linespace
\usepackage[a4paper, lmargin=0.1666\paperwidth, rmargin=0.1666\paperwidth, tmargin=0.1111\paperheight, bmargin=0.1111\paperheight]{geometry} %margins
%\usepackage{parskip}

\usepackage[all]{nowidow} % Tries to remove widows
\usepackage[protrusion=true,expansion=true]{microtype} % Improves typography, load after fontpackage is selected

\usepackage{lipsum} % Used for inserting dummy 'Lorem ipsum' text into the template


%-----------------------
% Set pdf information and add title, fill in the fields
%-----------------------
\hypersetup{ 	
pdfsubject = {},
pdftitle = {},
pdfauthor = {}
}

%-----------------------
% Begin document
%-----------------------
\begin{document} %All text i dokumentet hamnar mellan dessa taggar, allt ovanför är formatering av dokumentet

\section{Overview}
\begin{itemize}
    \item Dynamic Bonus framework in the spirit of EGSS to compare with Wage Rigidity.
    \item  We consider first the case of no savings with general utility and the case with savings for log-utility.
    \item The firm's output is $$y_{it}=z_t(a_{it}+\eta_{it})$$
    \item We make the effort after noise assumption as in EGSS, and we assume that a unique level of effort $a(z^t)$ is implemented irrespective of the idiosyncratic shock.
    \item We assume $\eta_{it}$ is a Markovian process.
    \item The value of a Firm's vacancy at time 0 is
    $$\pi_{i0}=E_0 \sum_{t=0}^{\infty}(\beta (1-s))^t[z_t(a_{it}+\eta_{it})-w_{it}]-\frac{\kappa}{q(\theta_0)}$$ 
    \item We also assume $\beta(1+r)=1$
   
\end{itemize}



\subsection*{Bonus Wage Economy}

\begin{itemize}
    \item The firm maximizes the value of filling a vacancy
    $$V(z_0)=\max_{\{a_t(z_t),w_{it}\}} E_0 \sum_{t=0}^{\infty}(\beta (1-s))^t[z_t(a_{it}+\eta_{it})-w_{it}]$$
    subject to the incentive compatibility constraint wich requires that for any alternative effort strategy $$\sum_{t=0}^{\infty} \int \int (\beta (1-s))^t [u(w_t(a_t,\eta^t|z^t)-h(a_t(z^t))+\beta s \omega(z_{t+1})]\pi(\eta_{it})\pi(z_t)d\eta_{it}dz_t$$
     $$\geq \sum_{t=0}^{\infty} \int \int (\beta (1-s))^t [u(w_t(\tilde{a}_t,\eta^t|z^t)-h(\tilde{a}_t(z^t,\eta^t))+\beta s \omega(z_{t+1})]\pi(\eta_{it})\pi(z_t)d\eta_{it}dz_t$$ for any alternative effort strategy $\{\tilde{a}_t(z^t,\eta^t))\}_{t=0,\dots,\infty}$ and the individual rationality (participation constraint)
     $$\sum_{t=0}^{\infty} \int \int (\beta (1-s))^t [u(w_t(a_t,\eta^t|z^t)-h(a_t(z^t))+\beta s \omega(z_{t+1})]\pi(\eta_{it})\pi(z_t)d\eta_{it}dz_t \geq \omega(z_0)$$
     \item Equilibrium: free entry, so the value of creating a vacancy in zero
     $$\pi_{i0}=0,\quad V(z_0)=\frac{\kappa}{q(\theta_0)}$$
\end{itemize}

\subsection*{Optimal Contract}
we are gonna start with the intertemporal properties of the optimal contract, then characterise in intratemporal properties

\begin{lemma}[Intertemporal property of the earnings process]
In the case without savings the Inverse Euler (IEE) equation holds
$$\frac{1}{u'(w_t(a_t,\eta^t|z^t)}=E[\frac{1}{u'(w_{t+1}(a_{t+1},\eta^{t+1}|z^{t+1})}|\eta^t,z^t]$$
In the case with savings and log utility the Euler equation holds and is:
$$E[\frac{w_t(a_t,\eta^t|z^t)}{w_{t+1}(a_{t+1},\eta^{t+1}|z^{t+1})}|\eta^t,z^t]=1$$
\end{lemma}

\begin{assumption} (Technical assumptions)
\begin{itemize}
    \item $\eta$ is distributed as $\mathcal{N}(0,\sigma^2_{\eta})$
    \item finite horizon $S$
    \item isoelastic disutility of effort with parameter $\varepsilon$
    
\end{itemize}
\end{assumption}
\begin{proposition}
Define the present value of output by 
$$Y=E_0 \Big[\sum_{t=0}^{S} (\beta (1-s))^t z_t a_{t} \Big]$$ and the sequence of pass-through parameters $$\psi_t=\frac{1}{\sum_{k=0}^{S-t}(\beta (1-s))^{k}}$$. In infinite horizon, the pass-through is constant equal to 
$$\psi=\frac{1}{1-\beta (1-s)}$$
The earnings schedule satisfies (without savings)
$$\log(w_t(a_t,\eta^t|z^t))=\log(w_{t-1}(a_{t-1},\eta^{t-1}|z^{t-1}))+\psi_t h'(a_t)\eta_t-\frac{1}{2}(\psi_t h'(a_t)\sigma_{\eta})^2$$
and with savings
$$\boxed{\log(w_t(a_t,\eta^t|z^t))=\log(w_{t-1}(a_{t-1},\eta^{t-1}|z^{t-1}))+\psi_t h'(a_t)\eta_t+\frac{1}{2}(\psi_t h'(a_t)\sigma_{\eta})^2}$$
where the time zero earnings are $$w_0= \psi_1(Y-\frac{\kappa}{q(\theta_0)}).$$
The optimal effort level satisfies

$$a_t(z_t)=\Big[\frac{z_t a_t}{\psi_1(Y-\frac{\kappa}{q(\theta_0)})}-\psi_t\varepsilon_{\psi_t h'(a_t), a_t}(h'(a_t)\sigma_{\eta})^2\Big]$$
and with savings

\end{proposition}
\begin{proof}[Proof]
Starting from an incentive compatible allocation, consider the following variations in retained wage/utility:
$$\hat{u}_{t}=u(w_t(a_t,\eta^t|z^t)+\beta s \omega(z_{t})-\beta(1-s) \Delta$$
$$\hat{u}_{t+1}=u(w_{t+1}(a_{t+1},\eta^{t+1}|z^{t+1})+\beta s \omega(z_{t+1})+ \Delta$$ These perturbations preserve utility and incentive compatibility since for all $a_t$
$$\hat{u}_t - h(a_t)+\beta(1-s)E[\hat{u}_{t+1}|\eta^t,z^t]$$
$$=u(w_t(a_t,\eta^t|z^t)-h(a_t(z^t))+\beta s \omega(z_{t+1})+\beta(1-s)E[u(w_t(a_{t+1},\eta^{t+1}|z^{t+1})+\beta s \omega(z_{t+1})|\eta^t,z^t]$$. Replacing this in the firm's problem we get the result.

We provide a heuristic proof of this proposition, and the formal argument follows the same steps as in EGSS. Assume that a unique level of effort is implemented at each time t, that these effort levels are independent of previous output noise, and that local incentive constraints are sufficient conditions. Consider first the incentive compatibility constraint which ensures that the worker does not wish to choose a different level of effort than the one recommended by the firm. Consider a local deviation in effort at after history $(\eta^t,z^t)$. The effect of such a deviation on the worker's lifetime utility U should be zero,
$$E_{t-1}[\frac{\partial U}{\partial y_t}\frac{\partial y_t}{\partial a_t}+\frac{\partial U}{\partial a_t}|z_t]=0$$.
Since $\frac{\partial y_t}{\partial a_t}=z_t$ then
$$E_{t-1}[\frac{\partial U}{\partial y_t}|z_t]=-\frac{1}{z_t}[\frac{\partial U}{\partial a_t}|z_t]$$
\end{proof}

Note that we can obtain the optimal contract in closed form with
\begin{itemize}
    \item private savings and log utility, main change is $+\frac{1}{2}(\psi_t h'(a_t)\sigma_{\eta})^2,\dots$
    \item AR(1) prices for $\eta$ and z
\end{itemize}

\subsection*{Case with Savings}
Applying the IC for effort above in the final perdiod, we obtain
$$u'(w_S(a_S,\eta^S|z^S))\frac{\partial w_S(a_S,\eta^S|z^S)}{\partial \eta_S}=h'(a_S(z^S))$$
Fixing $\eta^{S-1}$ and integrating this incentive constraint over $\eta_S$ ((meaning over realizations of $\eta_S$ given $a_S(z^S)$) leads to
$$u(w_S(a_S,\eta^S|z^S))=h'(a_S(z^S))\eta_S+g^{S-1}(\eta^{S-1})$$ for some function of past output $g^{S-1}(\eta^{S-1})$. This implies in particular that
$$\frac{\partial u(w_S(a_S,\eta^S|z^S))}{\partial \eta_{S-1}}=\frac{\partial g^{S-1}(\eta^{S-1})}{\partial \eta_{S-1}}$$
Analogously, the incentive constraint for effort in the second to last period reads
$$u'(w_{S-1}(a_{S-1},\eta^{S-1}|z^{S-1}))\frac{\partial w_{S-1}(a_{S-1},\eta^{S-1}|z^{S-1})}{\partial \eta_{S-1}}+\beta (1-s)u'(w_S(a_S,\eta^S|z^S))\frac{\partial w_S(a_S,\eta^S|z^S)}{\partial \eta_{S-1}}=h'(a_{S-1}(z^{S-1}))$$
Integrating the previous expression over $\eta_{S-1}$ and using the previous equation implies
$$u(w_{S-1}(a_{S-1},\eta^{S-1}|z^{S-1}))+ \beta(1-s)g^{S-1}(\eta^{S-1})=h'(a_{S-1}(z^{S-1}))\eta_{S-1}+g^{S-2}(\eta^{S-2})$$
We now want to show that $g^{S-1}(\eta^{S-1})$ is a linear function of $\eta_{S-1}$. Since the utility
function is logarithmic we have:
$$\log (w_S(a_S,\eta^S|z^S))=h'(a_S(z^S))\eta_S+g^{S-1}(\eta^{S-1})$$ and
$$\log (w_{S-1}(a_{S-1},\eta^{S-1}|z^{S-1})) =h'(a_{S-1}(z^{S-1}))\eta_{S-1}-\beta(1-s)g^{S-1}(\eta^{S-1})+g^{S-2}(\eta^{S-2})$$
We add the part of the proof that is specific to hidden savings. Recall that the Euler equation reads:
$$E_{S-1}[\frac{1}{w_{S}(a_{S},\eta^{S}|z^{S})}|\eta^{S-1},z^{S-1}]=\frac{1}{w_{S-1}(a_{S-1},\eta^{S-1}|z^{S-1})}$$
Using the previous expressions, this equality can be rewritten as
$$E_{S-1}[e^{-h'(a_S(z^S))\eta_S}|\eta^{S-1},z^{S-1}]e^{-g^{S-1}(\eta^{S-1})}=e^{-h'(a_{S-1}(z^{S-1}))\eta_{S-1}+\beta(1-s)g^{S-1}(\eta^{S-1})-g^{S-2}(\eta^{S-2})}$$
This in turn implies
$$(1+\beta(1-s))g^{S-1}(\eta^{S-1})=\frac{1}{2} (h'(a_S(z^S))\sigma_{\eta})^2+h'(a_{S-1}(z^{S-1}))\eta_{S-1}+g^{S-2}(\eta^{S-2})$$
Therefore $g^{S-1}(\eta^{S-1})$ and in turn $u(w_{S-1}(a_{S-1},\eta^{S-1}|z^{S-1}))$ is linear in $\eta_{S-1}$ Moreover, the last-period utility is linear in both $\eta_{S}$ and $\eta_{S-1}$. By induction, we can show that the utility in each period is a linear function of the performance shock in every past period. Now suppose for simplicity of exposition that $S = 2, \beta = 1, r = 0, s=0$. From the argument above
 = 1, so that 1 = 1
2 and 2 = 1. From the arguments above we have a log-linear
specification for earnings:
$$\log w_1 = \alpha_1\eta_1+k_1$$
$$\log w_2 = \alpha_{21}\eta_1+\alpha_{22}\eta_2+k_1+k_2$$
The Euler equation is so that
$e^{-\alpha_1\eta_1-k_1}=e^{-\alpha_{21}\eta_1-k_1}E[e^{-\alpha_{22}\eta_2-k_2}|\eta_1]$. This requires that $\alpha_1=\alpha_{21}$ and $e^{k_2}=E[e^{-\alpha_{22}\eta_2}|\eta_1]$
Writing down the utility function we obtain 
$$\alpha_1=\frac{1}{2} h'(a_1) \quad \alpha_2= h'(a_2)$$
From this we conclude (a few steps) that
$$\log(w_t(a_t,\eta^t|z^t))=\log(w_{t-1}(a_{t-1},\eta^{t-1}|z^{t-1}))+\psi_t h'(a_t)\eta_t+\frac{1}{2}(\psi_t h'(a_t)\sigma_{\eta})^2$$
\section{Two Output Model}

\begin{itemize}
\item Aggregate shock $\theta$
\vspace{10pt}
\begin{itemize}
\item preferences $v (c) - h (\ell)$ in consumption $c$ and labor effort $\ell \in [0;1]$

\end{itemize}
\item Worker who provides effort $\ell$ produces $\theta$ units with probability $\pi(\ell)\underbrace{=}_{\text{wlog}}\ell$, 0 otherwise
\vspace{10pt}
\begin{itemize}
\item \textcolor{blue}{moral hazard}: firm observes worker's ability and output, but not effort.
\item contract:
\vspace{10pt}
\begin{itemize}
\item effort $\ell(\theta)$,
\item earnings: base (or fixed) salary $\textcolor{blue}{\underline{z}(\theta)}$, high pay $\textcolor{blue}{\bar{z}(\theta)}$. i.e bonus $\textcolor{blue}{b(\theta)\equiv\bar{z}(\theta)-\underline{z}(\theta)}$.
\end{itemize}
\item key is bonus rate $\beta(\theta)$ defined by $\textcolor{blue}{\exp^{\beta(\theta)}\equiv\bar{z}(\theta)/\underline{z}(\theta)}$
\vspace{10pt}
\begin{itemize}
\item $\textcolor{blue}{\beta(\theta)	>0}$: incomplete insurance against output risk within
the firm for effort incentives.
\end{itemize}
\end{itemize}
\end{itemize}



\begin{itemize}
\item Firm hires worker $\theta $ and maximizes \textcolor{blue}{expected profits} given taxes and reservation value $U(\theta)$
\begin{equation}\label{eq:profit}
  \textcolor{blue}{\Pi(\theta)} \ = \ \max_{\{\underline{z},\bar{z}\}} \hspace{0.25em} \ell\theta - \mathbb{E}[z],
\end{equation}
where $ \mathbb{E}[z] = (1-\ell)\underline{z} + \ell \bar{z}$.
\item \textcolor{blue}{Incentive constraint}: contract must induce the worker to provide effort $\ell$
\begin{equation}\label{eq:incentive-global}
  \ell \ = \ \arg\max_{l \in [0,1]} \hspace{0.25em} (1-l)v(\underline{z}) + l v(\bar{z}) - h(l).
\end{equation}
\item \textcolor{blue}{Participation constraint}: minimum required utility is the reservation value
\begin{equation}\label{eq:participation}
  \mathbb{E}[v(z)] - h(\ell) \ \ge \ U(\theta),
\end{equation}
where $\mathbb{E}[v(z)] = (1-\ell)v(\underline{z}) + \ell v(\bar{z})$. 
\item \textcolor{blue}{Free-entry} (zero profits) in labor market $\theta$ pins down reservation value $\textcolor{blue}{U(\theta)}$ in equilibrium.
\end{itemize}



\begin{itemize}
\item \textcolor{blue}{Incentive constraint pins down the optimal amount of risk (bonus)} to which the firm exposes the worker in order to elicit an effort level $\ell$.
\vspace{10pt}
\begin{itemize}
\item necessary and sufficient condition for incentive compatibility
\begin{equation}\label{eq:incentive-local}
  h'(\ell) \ = \ v(\bar{z}) - v(\underline{z})
  \end{equation}
\item assume: log utility then
\begin{equation}\label{eq:bonus rate-CRP}
 \boxed{ \textcolor{blue}{\beta(\theta) \ = \ h'(\ell(\theta)).}
\end{equation}
\end{itemize}
\item The IC bonus rate \textcolor{blue}{ $\beta(\theta)$ is increasing in $\ell(\theta$)} since the disutility of effort is convex.
\vspace{10pt}
\begin{itemize}
\item moral hazard: eliciting higher effort requires stronger incentives (risk)
\end{itemize}
\end{itemize}



\begin{itemize}
\item Earnings
\begin{equation}\label{eq:earnings-CRP}
  \underline{z}(\theta) \ = \ \frac{1}{1+\ell (e^\beta - 1)}\ell\theta \quad \text{ and }\quad \bar{z}(\theta) \ = \ \frac{e^\beta}{1+\ell (e^\beta - 1)}\ell\theta.
\end{equation}
\vspace{10pt}

\item Labor effort satisfies
\begin{equation}\label{eq:effort-CRP}
  \theta \ = \ b \Big[ 1 + \underbrace{\frac{1}{1-p} \ell (1-\ell) h''(\ell)}_{\text{MCI}} \Big].
\end{equation}
%\vspace{10pt}
\begin{itemize}
\item Keepings earnings $\underline{z},\bar{z}$ fixed, higher effort costs $b$ to the firm. 
\item In addition,  firm needs to raise the spread in earnings $\beta$ for extra effort incentive: 
\\ \textcolor{blue}{marginal cost of incentives $\propto h^{''}(\ell)$}
\end{itemize}
\end{itemize}

\end{document}
